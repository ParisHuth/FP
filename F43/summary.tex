\section{Zusammenfassung}
In diesem Versuch haben wir uns mit der Raman-Spektroskopie beschäftigt, um die Eigenschaften von Molekülen wie Deuterium, ortho- und para-Wasserstoff, Sauerstoff und Stickstoff zu untersuchen.
Der Versuch begann mit der Charakterisierung des Raman-Kantenfilters, bei dem die Abhängigkeit einer Photodiodenspannung von dessen Einfallswinkel gemessen wurde. Dabei wurden die Cut-on-Winkel von $\vartheta_1 = \left(-8.18\pm 0.17\right)^\circ$ und $\vartheta_2 =\left(10.36\pm 0.21\right)^\circ$ bestimmt. 
Während dieser Messungen traten jedoch Probleme mit der Laserstabilität auf, die dazu führten, dass die Intensität des Lasers stark abnahm und keine weitern Messungen zun\"achst möglich waren. % Might cut later

Im nächsten Schritt wurde die PMT-Kennlinie untersuch, um die Abhängigkeit der Zählrate von der angelegten Beschleunigungspannung $U_B$ zu bestimmen. Hier zeigte sich, dass im Bereich von $1.8-2.0$kV die Zählrate linear von der Spannung abhängt. Als optimale Betriebsspannung wurde $U_B´=2.0$kV gewählt, um das Signal-to-Noise-Ratio zu maximieren. Allerdiengs traten erneut Probleme mit der Laserstabilität auf, die zu fehlerhaften Messungen führten. Da das Gerät nich rechzeitig repariert werden konnte, wurde ältere Messwerte des Betreuers verwendet, um die Auswertung durchzuführen.

Der Hauptteil des Versuchs bestand in der Untersuchung der Raman-Spektren verschiedener Moleküle. Zunächst wurde das Rotationsspektrum von Deuterium analysiert. Es konnten vier Raman-Linien identifiziert werden, die den ersten vier Rotationsanregungen entsprachen. Die experimentell bestimmte Rotationskonstante $B_{exp}^{D_2} = \left(28.5\pm 0.5\right)\mathrm{cm}^{-1}$, wich jedoch signifikant vom Literaturwert abwich $B_{lit}^{D_2} = 30.57\mathrm{cm}^{-1}$. Auch die reduzierte Masse $\mu_{exp}^{D_2} = \left(1.079\pm0.019\right)$u zeigt eine Abweichung vom erwarteten Wert.

Anschlie{\ss}end wurden die Ramen-Spektren von ortho- und para-Wasserstoff untersucht. Dabei zeigte sich die erwarteten Rotationsanregung, und die Reinheit der para-Wasserstoff-Probe wurde auf etwa $96.2\%$ gesch\"atzt. Die experimentell bestimmten Rotationskonstanten $B_{exp}^{H_2} = \left(57.8\pm0.3\right)\mathrm{cm}^{-1}$ (normal) und $B_{exp}^{H_2} = \left(57.8\pm0.9\right)\mathrm{cm}^{-1}$ (para) wichen ebenfalls vom Literaturwert $B_{lit}^{H_2} = 61.0841\mathrm{cm}^{-1}$ ab. Bei der Analyse der para-Wasserstoff-Probe wurde festgestellt, dass die Untergrundmessung nicht erfolgreich war, was zu einer Verschiebung des Spektrums f\"uhrte und die Genauigkeit der Fitparameter beeintr\"achtigte.

Schließlich wurden die Vibrations-Raman-Spektren von Sauerstoff und Stickstoff untersucht. Aufgrund der h\"oheren reduzierten Massen und des gr\"o{\ss}eren Atomabstands dieser Molek\"ule war die Rotationsstruktur nicht aufl\"osbar, jedoch konnte die erste Schwingungsanregung gemessen werden. Die gemessenen Schwingungsenergien betrugen $E_{vib} =\left(182.88\pm0.07\right)$meV f\"ur O$_2$ und $E_{vib} =\left(278.4\pm0.5\right)$meV f\"ur N$_2$. Diese Werte wichen jedoch stark von den erwarteten Werten ab, was auf m\"ogliche systematische Fehler in der Messapparatur oder Probleme bei der Untergrundkorrektur hindeutet.

Insgesamt zeigt der Versuch, dass die Raman-Spektroskopie eine leistungsstarke Methode zur Untersuchung molekularer Eigenschaften ist. Allerdings traten w\"ahrend der Messungen mehrere technische Schwierigkeiten auf, insbesondere die Instabilit\"at des Lasers und Probleme bei der Untergrundkorrektur, die die Genauigkeit der Ergebnisse beeintr\"achtigten. Trotz dieser Herausforderungen konnten die grundlegenden Raman-Linien identifiziert und die Rotations- und Schwingungsanregungen erfolgreich analysiert werden. Die signifikanten Abweichungen zwischen den experimentell bestimmten und den erwarteten werten deuten jedoch darauf hin, dass eine sorgf\"altige Kalibrierung und stabile Messbedingungen f\"ur zuk\"unftige Experimente entscheidend sind.
