\section{Measurements Procedure}
\subsection{Relaxation time}
In this part of the experiment we study the relaxation times $T_1$ and $T_2$ of samples GD500 and GD600. For that we first start by calibrating the magnet of the p20 unit by inserting the GD600 sample and setting the 90° pulse for it to maximize the received signal. Analogously the 180° is set up for it to minimize the NMR signal. 
After calibration the measurements are taken as explain above: For $T_2$ we use the Hahn echo method and Carr-Purcell metod, whereas for $T_1$ we implement a 180°-90° sequence. 
During the entirety of this measurement the working frequency was set to $\omega_w = (1 \pm 0.5)$ kHz
\subsubsection{Chemical shift}
Next we use the NMR to identify 5 substances. For the identification of the substances we used TMS as a reference substance, hence we had two samples for each susbtance: one for the substance alone and a second sample mixed with TMS.
The samples is inserted in the magnet of the p20 unit, then the probe was set in rotation with help of a air pump. This procedure was implemented to minimize the field inhomogeneities, hence minimazing the width of the signal in frequency space. Consequently we perform a spin-echo measurement. The signal is Fourier transformed and a Gauss profile is fitted into the different peaks of the spectrum to measure the frequency of the Maxima. 
The spectrum of the sample without TMS is compared to the one with TMS to identify the Maximum corresponding to the reference signal. Finally the relative distance between both Maxima is measured. 
\subsubsection{Imaging}
The last part of the experiment consist in 3 small tasks. All of them were performed using a Bruker minispec mq7.5 directly connected to  computer.
\paragraph{1 dimensional imaging}
In this part we introduce three different samples and study their 1 dimensional profiles. 
\paragraph{Time evolution of system}
We prepare a test tube with sand. Oil is poured in the test tube and its quickly placed into the NMRI device. To study the time evolution of this test tube we take different  1 dimensional profiles throughout a time interval. 
\paragraph{2 dimensional imaging}
We place different objects in the NMRI machine and take 2D profile of them. 

