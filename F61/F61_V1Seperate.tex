\documentclass{article}
\usepackage{float}
\usepackage{graphicx} % Required for inserting images
\usepackage{fontenc}
\setmainfont{Open Dyslexic}

\title{F61 - Nuclear Magnetic Resonance}
\author{Paris J. Huth - Q'inich Figueroa}
\date{May 2024}

\begin{document}

\maketitle

\section{Introduction}
In Protocoll we will examine the usage nuclear magnetic resonance to identify probes and reveal the structure of objects.
\section{Basics}

Any nuclei with an existent spin $S$ has a magnetic dipole moment: 
$$ \vec{\mu} = \hbar \gamma \vec{S}$$
where $\gamma$ is the gyromagnetic ration. 

This magnetic dipole $\vec{\mu}$ interacts with an external magnetic field $\vec{B}_0$ and is associated with an interaction energy $\Delta E$:

$$\Delta E = -\vec{\mu}\cdot \vec{B}_0$$

This interaction yields both a parallel $\mu_+$ and antiparallel $\mu_-$ orientation of the protons magnetic dipole in the external field. 
For a macroscopic sample of $N$ protons, the number of occupied states $N_+$ and $N_-$, the sum of which comprises $N$, can be approximated by a Boltzmann distribution:

$$ N_\pm = N_0e^{-\frac{E_0\pm\Delta E}{kt}}$$
with a normalization factor $N_0$.

\section{Measurements}
\section{Analysis}
\section{Critical Discussion}
\end{document}
