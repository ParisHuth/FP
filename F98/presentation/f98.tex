\documentclass[16pt]{beamer}
\usepackage[utf8]{inputenc}
\usepackage[T1]{fontenc}
\usepackage{lmodern}
\usetheme{Madrid}
\begin{document}
	\author{Figueroa, Q. \& Huth, P. J.}
	\title{F98}
	\subtitle{SQUIDs and Noise Thermometers}
	%\logo{}
	%\institute{}
	%\date{}
	%\subject{}
	%\setbeamercovered{transparent}
	%\setbeamertemplate{navigation symbols}{}
	\begin{frame}[plain]
		\maketitle
	\end{frame}
	
	% TODO replace for actual content
	\begin{frame}
		\frametitle{Structure}

		\begin{itemize}
			\item Physikalische Fragestellung 
			\begin{itemize}
				\item SQUIDs
				\item Noise Thermometers
				\item kurze theoretische Einleitung
			\end{itemize}
			\item Messprinzip und Apparatur 
			\item Vorstellung der Messergebnisse und deren Auswertung
			\item kritische Diskussion der Ergebnisse
			\item Anmerkung zum Versuch
		\end{itemize}
	\end{frame}

	\begin{frame}
		\frametitle{Physics in Question}
		\begin{itemize}

			\item Question: How to measure the temperature of a system using noise.
			\item Answer: Using a SQUID.\\
				Very sensitive magnetometers, consisting of superconducting loops interrupted by Josephson junctions.
		\end{itemize}
	\end{frame}

	\begin{frame}
		\frametitle{What we measure and why}
		\begin{enumerate}
			\item Resistance at room temperature and in liquid Helium:\\
			Observe the change in Resistence
			\item Current - Voltage characteristics $V-I$: \\
			Estimate critical current $I_C$
			\item Measure the $V-\Phi$ Characteristics:\\
			Determine the inverse mutual inductance $ M_{IN}^{-1}$ and $M_{\Phi B}^{-1}$
			\item Measure output resulting from  periodic input signal: \\
			Determine the amplification
			\item Measure Noise at different GBP:\\
			Finding an optimal value for the GBP
			\item Measure Noise Spectrum with a two stage SQUID: \\
			Calculate the temperature
			
		\end{enumerate}
	\end{frame}
		
	% -----------------
	\begin{frame}
		\begin{columns}
			% Left column for the list
			\begin{column}{0.4\textwidth}
				\begin{itemize}
					% Superconductivity and its sub-items
					\item<1-2> Superconductivity
					\begin{itemize}
						\item<2> Meissner-Ochsenfeld effect
					\end{itemize}
					
					% Josephson junction and its sub-items
					\item<3-5> Josephson junction
					\begin{itemize}
						\item<4> Flux quantization
						\item<5> Cooper pairs
					\end{itemize}
					
					% SQUIDs and its sub-items
					\item<6-9> SQUIDs
					\begin{itemize}
						\item<7> DC SQUID
						\item<8> Flux locked loop
						\item<9> Two-stage SQUID
					\end{itemize}
				\end{itemize}
			\end{column}
			
			% Right column for corresponding content
			\begin{column}{0.6\textwidth}
				\only<1>{
					\textbf{Superconductivity:}
					\par Expulsion of magnetic fields from a superconductor below its critical temperature.
				}
				\only<2>{
					\textbf{Meissner-Ochsenfeld Effect:}
					\par Expulsion of magnetic fields from a superconductor below its critical temperature.
				}
				\only<3>{
					\textbf{Josephson junction:}
					\par Josephson junction
				}
				\only<4>{
					\textbf{Flux Quantization :}
					\par Flux quantization plays a role in the phase relationship between the two superconductors.
				}
				\only<5>{
					\textbf{Cooper Pairs:}
					\par Cooper pairs tunnel through the insulating barrier, causing the Josephson effect.
				}
				\only<6>{
					\textbf{SQUIDs:}
					\par SQUIDs.
				}
				\only<7>{
					\textbf{DC SQUID:}
					\par A superconducting quantum interference device with two Josephson junctions for measuring magnetic flux.
				}
				\only<8>{
					\textbf{Flux Locked Loop:}
					\par A feedback loop that stabilizes the SQUID output by maintaining constant magnetic flux
				}
				\only<9>{
					\textbf{Two-Stage SQUID:}
					\par Enhances sensitivity by using a primary SQUID amplified by a secondary stage.
				}
			\end{column}
		\end{columns}
	\end{frame}
		
		
	\begin{frame}
		\frametitle{Takeaways}
		\begin{itemize}
			\item dc-SQUIDs can be very sensitive ampfiliers for high precision applications such as:
			\begin{itemize}
				\item Measuring the CMB in the COBE \& PLANCK Satelites.
			 \end{itemize}
		\end{itemize}
	\end{frame}

\end{document}